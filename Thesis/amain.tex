%% This document gives an example on how to use the ntnubachelorthesis
%% LaTeX document class.
%% Use oneside for PDF delivery and twoside for printing in a book style
%% use language english, norsk, nynorsk and one of the following shortenings
%%  ``BSP'' Bachelor i Spillprogrammering,\\
%%  ``BRD'' Bachelor i drift av nettverk og datasystemer,\\
%%  ``BIS'' Bachelor i Informasjonssikkerhet,\\
%%  ``BPU'' Bachelor i Programvareutvikling, \\
%%  ``BIND'' Bachelor i Ingeni�rfad - data. 
\documentclass[BRD,norsk,oneside]{other/ntnubachelorthesis}

\usepackage{csvsimple}
\usepackage{booktabs}
%\usepackage[english,nynorsk]{babel}
\usepackage[toc]{glossaries}
%\usepackage[acronym,nomain]{glossaries}

%\newcommand{\comment}[1]{\textcolor{blue}{\emph{#1}}}  %% use of the colour and you can see how to use commands with parts \comment{so what}

%% The class files defines these two
%% \newcommand{\NTNU}{Norwegian University for Science and Technology} %
\newcommand{\TMX}{Telemix AS }

% you can create you one #define like structures using the \newcommand feature
% you can change behaviour using \renewcommand

\newcommand{\com}[1]{{\color{red}#1}} % supervisor comment
%\renewcommand{\com}[1]{} %remove starting % to remove supervisor comments
% This will appear in text \com{Lecuters comment} and be visible unless you uncomment
% the renewcommand line.

\newcommand{\todo}[1]{{\color{green}#1}} % items to do
%\renewcommand{\todo}[1]{} %remove starting % to remove items to do

\newcommand{\n}[1]{{\color{blue}#1}} % other comment
%\renewcommand{\n}[1]{} %remove starting % to remove notes

\newcommand{\dn}[1]{} % add the d to a note to say that you have finished with it.


% Norwegian Characters,  needs the {} or to be separate from the next letters
% \o{}   \aa{}   \ae{}   so at the end of a word you can use \o  \aa   \ae
% \O{}   \AA{}   \AE{}   you can also just leave a space and latex will remove it
%                        eg, NTNU i Gj\o vik  or NTNU i Gj\o{}vik


%\selectlanguage{english}

%\selectlanguage{bokmal}

\makeglossaries
\begin{document}
%\newacronym{dns}{DNS}{Domain Name Server}
\newacronym{icmp}{ICMP}{Internet Control Message Protocol}
\newacronym{snmp}{SNMP}{Simple Network Management Protocol}
\newacronym{ram}{RAM}{Random Access Memory}
\newacronym{dhcp}{DHCP}{Dynamic Host Configuration Protocol}
\newacronym{ssh}{SSH}{Secure Shell}
\newacronym{http}{HTTP}{Hypertext Transfer Protocol}
\newacronym{https}{HTTPS}{Hyper Text Translate Protocol Secure}
\newacronym{cli}{CLI}{Command Line Interface}
\newacronym{cpu}{CPU}{Central Processing Unit}
\newacronym{rest}{REST}{Representational State Transfer}
\newacronym{api}{API}{Application Programming Interface}
\newacronym{lts}{LTS}{Long Term Support}
\newacronym{ubnt}{UBNT}{Ubiquiti Networks}
\newacronym{vlan}{VLAN}{Virtual Local Area Network}
\newacronym{lan}{LAN}{Local Area Network}

\thesistitle{TheMonMan, effektiv monitorering av varierte distribuerte systemer.}
\thesisshorttitle{TheMonMan} % use this if you have a very long title and want something shorter on the header pages
\thesisauthor{Glenn-Tore Jakobsen}
\thesisauthorA{John-Arvid Kibsgård}
%\thesisauthorB{}
%\thesisauthorC{}
\thesissupervisor{Eigil Obrestad}
%\thesissupervisorA{} %second supervisor

\nmtkeywords{Thesis, Latex, IMT}
\nmtdesc{TheMonMan vil løse utfordringer \TMX har for å få beskjed om når, hva og hvor noe er galt.}

\nmtoppdragsgiver{\TMX}
\nmtcontact{Helge Mortensen, helge@telemix.no, 75126000}




\thesisdate{\ntnubachelorthesisdate}
\useyear{16.05.2017}

\nmtappnumber{} %numebr of appendixes
\nmtpagecount{} %currently auto calculated but might be wrong




\thesistitleNOR{TheMonMan}
\nmtkeywordsNOR{Monitorering, Open Source, Distribuerte-systemer}
\nmtdescNOR{\TMX har mange distribuerte systemer spredt ut over Helgeland, noe som gir utfordringer med monitorering. Her skal TheMonMan være et aktivt og passivt system for monitorering. Monitorerings-systemet må tilpasses hvert enkelt scenario og det må gjøres på et modulært vis slik at det skal være enkelt å bygge opp.}

 % this is the file which contains all the details about your thesis

\makefrontpages % make the frontpages
\chapter*{Innledning} %the * means do not give the chapter a number
\label{chap:preface}

Vi ønsker å takke \TMX for deres komplekse system som krever denne form for monitorering. Uten dette ville det bare vært en vanlig monitorerings-oppgave
 
 


\tableofcontents
%\listoffigures
%\listoftables
\printglossaries



\input{tex/introduksjon}
%\input{requirements}
%\input{technical}
%\input{process}
%\input{implementation}
%\input{deployment}
%\input{testing}
\chapter{Script for automasjon og andre oppgaver}
\label{chap:script}
\lstset{language=bash}

For å få Zabbix til å sende en SMS via et REST API til clxcommunication.com må dette scriptet brukes. 
\begin{lstlisting}
#!/bin/bash

TOKEN=$1
FROM=$2
TO=$3
MESSAGE=$4
SERVICE=$5

#Send til API hos clxcomm

DATA="$( printf '{"from": "%s","to": ["%s"],"body": "%s"}' "$2" "$3" "$4" )"

curl -X POST \
         -H "Authorization: Bearer $TOKEN" \
         -H "Content-Type: application/json" \
         -d "$DATA" "https://api.clxcommunications.com/xms/v1/$SERVICE/batches"

\end{lstlisting}
    

%\input{discussion}
%\input{conclusion}
\chapter{Midlertidig skriv, flyttes senere}
\label{chap:midlertidig}

% Brukes for å få tankebokser
\newtcolorbox{tanke}[1]{title=#1}


\section{DNS}


For å få tilgang til monitoreringstjenesten utenfra vil det bli åpnet nødvendige porter i brannmur. For å gjøre det enklere mtp bytte av ip-adresse vil det brukes et A-record på DNS server. Denne skal være: themonman.telemix.no
Da vil alle klienter utenfra nå monitoreringsserver på themonman.telemix.no, og alle klienter internt når det fra themonman.telemix.local.


\section{Portåpning}

Hvis klienter skal ha tilgang utenfra må enkelte porter åpnes i brannmuren, siden vi velger å ha den bak en brannmur.
Disse portene vil være:
\begin{itemize}
    \item Port 80: for http (blir redirekted til https)
    \item Port 443: for https
    \item Port 10050: for zabbix passive agent
    \item Port 10051: for zabbix aktiv agent
\end{itemize}


\section{SMS}

For å gi beskjed til ansvarlige må det sendes sms, epost og andre tjenester vil ikke være like responsivt.
For å få Zabbix til å sende sms må Telemix betale for en tjeneste der et API kan brukes. Da kan Zabbix bruke et script for å sende ut sms til relevante personer.
SMS vil bare bli brukt for kritiske beskjeder.
Nexmo.com er en mulig tjeneste, hvis ikke Telemix som Telenor-forhandler har et annet alternativ.

Ref samtale over epost, Telemix ønsker en tjeneste de kan bruke til andre utsendelser også. Nexmo støtter bare et enkelt REST API, og selv om det er mulig å lage en løsning ut av dette er det mer aktuelt å finne en annen tjeneste som har flere muligheter. Da blir det gitt en sms løsning som ikke bare brukes til varsling av monitorering, men kan da brukes av Telemix i flere sammenhenger.





\section{Zabbix Server}

Zabbix server blir innstallert på en virtuell server i Telemix sitt Xen-miljø. Backup og redundans på servernivå vil ikke være aktuellt å implementere, men det må bekreftes hvordan det fungerer.


\section{Discovery}
For å gjøre det litt enklere og få inn hoster kan man bruke discovery, den vil scanne nettverket mellom et gitt intervall. Denne informasjonen kan man da lage “Actions” for, for eksempel å legge til disse nye enhetene som er funnet, eller scanne disse etter mere informasjon, eller “disable” sjekker hvis en vanlig klient ikke svarer på ping lenger (når bruker skrur av maskinen eller tar den med ut).

\begin{tanke}{One-Time Discovery}
Finne enheter i et nytt nettverk for å få den initielle samlingen av klienter.
\\
\\
"Vil ikke være nødvendig, hvis man skriver en discovery regel godt nokk vil den kunne leve sammen med systemet hele tiden."
\end{tanke}

\subsection{Switch Discovery}
For å få monitorert alle switcher som er i nettverket til Telemix må flere aspekter tas hensyn til. 
\begin{itemize}
    \item Hvilken form for monitorering skal det være, og hva støttes av switchene 
        \begin{itemize}
            \item SNMP v1, v2c eller v3
            \item ICMP
            \item SSH
            \item Telnet
        \end{itemize}
    \item Sikkerhet
    \item Standarder i forhold til leverandør
\end{itemize}

\begin{tanke}{Sikkerhet og muligheter}
    I forhold til sikkerhet rundt monitorering av switcher i nettverk, må man passe på at informasjon som flyter mellom systemene ikke blir tilgjengelig. (Brukernavn, passord, konfigurasjon)
    Telnet og SNMP v1/v2c tilbyr ikke noen form for kryptering av data, så alt kan dumpes og leses i klartekst. SNMP v3 støttes dog ikke av alle switchene i nettverket, og ICMP gir lite informasjon om enheten.
    SSH støttes heller ikke av alle enhetene så det vil ikke være mulig.
    
    "Siden Telemix kjører et eget VLAN der monitorering foregår er det kun ønskelige klienter på dette nettverket. For å få informasjon fra alle enheter er man nødt til å bruke usikre metoder, men pakke det inn i sikre miljø."
\end{tanke}

\section{Prioritering av monitorering}

Hva er viktigst å få monitorert, hva må prioriteres først.

\begin{itemize}
    \item Server
    \begin{itemize}
        \item Domene-Server
        \item DHCP-Server
        \item DNS-Server
        \item Epost-Server
        \item Web-Server
        \item Lagrings-Server
    \end{itemize}
    \item Klienter
    \begin{itemize}
        \item Kunder
        \item Interne
    \end{itemize}
    \item Nettverk
    \begin{itemize}
        \item Brannmur
        \item Node-Switcher
        \item Kjerne-Switcher
        \item Eksterne-Switcher
        \item Hoved-linker, trådløse
        \item Kunde-linker, trådløse
    \end{itemize}
    \item Zabbix (Seg selv)
\end{itemize}

Her må informasjon fra Telemix gis om hva som er viktigst for dem som bedrift. 

\section{Zabbix og Ubiquiti}
Telemix har mange Ubiquiti enheter som trenger monitorering, de består av linker, switcher og routere. 
Zabbix Share har en template liggende ute spesifikt for UBNT Airfiber. Denne templates burde kunne bygges på og endre for å dekke flere produkter.
\\\url{https://share.zabbix.com/network_devices/airfiber-radios}


\section{Zabbix og PFSense}
Telemix bruker hovedsaklig PFSense som sin brannmur-løsning. PFSense har som en vanlig brannmur mange forskjellige data som kan hentes inn og monitoreres, både til feilsøking og varsling.
Siden PFSense er bygd på FreeBSD vil det være muligheter for å eventuellt bruke en PFSense enhet som er Zabbxi Proxy.

Zabbix agent install PFSense:
\begin{enumerate}
    \item System -> Package Manager -> Available Packages
    \item Search for zabbix
    \item Install zabbix agent
    \item Services -> Zabbix Agent LTS
    \item Input data that is needed
\end{enumerate}


La til host i zabbix og la til freebsd template. 

\begin{tanke}{OBS}
    Man trenger å legge inn riktig IP og Hostname på Server og Agent. Utenom det vil alt fungere så lenge det ikke er satt restriksjoner på å sende ut data fra Agent.
\end{tanke}


\section{Zabbix og XEN-server}
Telemix bruker Citrix Xen-server til sitt virtuelle miljø. Disse servere er kritiske å monitorere, blir det noe feil med disse vil alle virtuelle instanser bli påvirket.
Xen-Server kjører på CentOS, så det vil være mulig å innstallere en zabbix agent på denne, men å installere tredje-parts programvare på en xen-server er ikke anbefalt og vil bryte en hver support.
Et alternativ vil være å monitorere ved hjelp av SNMP.



\section{Metoder for monitorering}
ICMP/Ping, for enheter som tillater dette, siden alle nettverksenheter støtter dette [ref], vil det være en enkel sjekk om selve nettverksenheten klarer å motta pakken og svare på denne. Det vil da fortelle at ruten til denne lokasjonen er ok. Siden denne sjekken ikke forteller noe annet kan denne sjekken være en “lavmålssjekk”.

En annen type sjekk vil da være direkte mot en tjeneste på en klient, for eksempel å sjekke om en port er åpen, eller om klienten svarer med forventet informasjon ved sjekk mot en port.

Den mest avanserte metoden for å monitorere vil være å bruke agenter for å få informasjon, da kan man bruke Zabbix sin egen agent på de klienter som støtter denne, eller man kan skrive script som tar seg av sjekker direkte på klienten og gir informasjon tilbake til server.




\section{Monitorering av Switcher}
For å monitorere switcher kan det brukes en vanlig ICMP sjekk, men for å få mer informasjon om tilstand til switchen må det konfigureres for SNMP på switchen og server må be om informasjon fra denne tjenesten.



\section{SNMP}
SNMP er en protokoll som har tre forskjellige hovedversjoner, v1, v2 og v3. De nyere versjoner bygger mer på autentisering og verifisering.



\section{Templates}
Templates brukes for å gjøre oppsett av monitorering enklere. Lager man en god template trenger man bare å notere klient-navn og klient-ip så vil tjenesten finne ut hva som trengs å monitoreres. Dette vil kreve mye av de som skriver templaten, så det må hele tiden tas vurdering på hvor mye skal automatiseres.






\bibliographystyle{other/bibliografistil}
\bibliography{bibliografi}




\appendix %after this line all chapters will have leters instead of numbers
%\input{projectplan}
%\input{gantt}
\input{tex/meetinglog}
%\input{progressreviews}
%\input{worklog}


\end{document}
